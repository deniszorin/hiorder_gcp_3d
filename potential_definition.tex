\documentclass{article}


\usepackage[utf8]{inputenc} % allow utf-8 input
\usepackage[T1]{fontenc}    % use 8-bit T1 fonts
\usepackage{hyperref}       % hyperlinks
\usepackage{url}            % simple URL typesetting
\usepackage{booktabs}       % professional-quality tables
\usepackage{amsfonts}       % blackboard math symbols
\usepackage{nicefrac}       % compact symbols for 1/2, etc.
\usepackage{microtype}      % microtypography
\usepackage{amsmath}        % for \eqref and math environments
\usepackage{lipsum}
\usepackage{fancyhdr}       % header
\usepackage{graphicx}       % graphics
\graphicspath{{media/}}     % organize your images and other figures under media/ folder
\usepackage{xcolor}

\begin{document}
\section{3D smoothed offset potential for meshes}

We have a mesh with the set of faces $F$, edges $E$ and  vertices $V$.   Each face is triangular; we assume that all vertices of each face are distinct, and  at most one edge connects any two vertices.  In this case, faces is defined by a sequence of vertex indices $(v_0, v_1, v_2)$ and edges are defined by pairs of vertices $(v_0, v_1)$.  The mesh is an  edge-manifold, i.e., each edge is  shared by at most two faces. We assume that if two faces share an edge $(v_0, v_1)$,  in one face the vertices are enumerated, up to circular permutation as $(v_0, v_1, v_2)$, and in the other $(v_1,v_0,v_3)$, 
i.e., the order is reversed. This implies the mesh is orientable.  

We define the 3D potential following the pattern of the 2D definition.  We start with defining the potential for a triangular face, then define the edge potential as a complement of the face potential, and finally the vertex potential.   

Let $H^\alpha(z)$ be a smoothed Heaviside function transitioning from $0$ to $1$ on the segment $[-\alpha,\alpha]$. For a point $q$, let $P_e(q)$, $e=e_0,e_1, e_2$ be the projection of the point to the line of one of the edges of the triangle,  and $r^f(q)$ be the distance to the triangle plane. Let 
\[
\Phi^{e,f}(q) = (q-P_e(q))_+ \cdot (n \times d_e)
\], 
where $d_e$ is a unit vector along the edge, and 
$n\times d_e$ is the inward pointing perpendicular to the edge, assuming $e$'s direction is chosen to be CCW, with respect to the normal direction.

Then we define the potential for a triangle as 
\[
I^f_\alpha(q) = \frac{H^\alpha(\Phi^{e_0,f}(q))
H^\alpha(\Phi^{e_1,f}(q))H^\alpha(\Phi^{e_2,f}(q))}{r^f(q)^p} = \frac{B^f_\alpha(q)}{r^f(q)^p}
\]
The numerator is an angular blending function, which for
$\alpha = 0$ becomes the product of 3 Heaviside functions, i.e. 1, if  $(q-P_i(q))_+ \cdot (n \times e_i) > 0$, for all i, and zero otherwise, i.e., the projection of $q$ is in the interior of the triangle. 

\paragraph{Edge potential.} Suppose an 
edge $e$ is shared by triangles $f_0$ and $f_1$, and 
let $P_e(q)$ be the projection of $q$ on the edge; note that this is the same point that is used in $\Phi_i$ computation for the incident triangles. 

In addition, define 
\[
\Phi^{j,e}(q) = (q - p_j)_+ \cdot (p_k - p_j)_+
\]
where $(j,k) = (0,1), (1,0)$, and $p_0$, $p_1$ are the two endpoints of the edge.
Note that $(p_k - p_j)_+$ are the vectors $d_e$ for the two incident faces. 

% DZ this version has the problem that if 
% the potential extends > triangle size the edge potential 
% support may have disconnected components "on the other sides" of adjacent triangles, where $B^f$ vanishes. 
%\[
%I^e_\alpha(q) = \frac{ (1- B^{f_0}(q) - B^{f_1}(q))H^\alpha(\Phi^{0,e}(q))H^\alpha(\Phi^{1,-e}(q))}{r^e(q)^p} \]

\[
I^e_\alpha(q) = \frac{ (1- H^\alpha(\Phi^{e,f_0}(q))-H^\alpha(\Phi^{e,f_1}(q))H^\alpha(\Phi^{0,e}(q))H^\alpha(\Phi^{1,-e}(q))}{r^e(q)^p} \]
where  $\Phi^{e,f_i}$ are computed as in the face computation, and $-e$ indicates that the edge orientation is switched. 

\paragraph{Vertex potential.} For vertices, we define the potential 
with respect to all incident faces and edges.
$F_v$ be the set of incident triangles, and $E_v$ the set of incident edges.  Denote $e_i(v,f)$, $i=0,1$, for a face $f$ and vertex $v$ to be the two edges (in any order) that have endpoint $v$ and are edges of face $f$.  Denote $f_1(e)$ and $f_0(e)$ to be the two faces sharing the edge $e$ (for a boundary edge, only one is defined, and in the formula below, it is skipped). $\Phi^{e_i(v,f),f}$ and $\Phi^{e,f_1(e)}$
are per (edge, face) pair quantities computed for the face potential. 
For an edge $e$ with endpoints $p_0$ $p_1$, $\Phi^{v,e} = \Phi^{0,e}$ 
defined as for the edge potential,  if $v = p_0$ and $\Phi^{v,e} = \Phi^{1,-e}$, if $v = p_1$. 


% Same problem as with the edge simplest version
%
%\[
%I^v_\alpha(q) = \frac{ 1- \sum_{f \in F_v} B^f(q)}{r^v(q)^p}
%\]

Then
\[
I^v_\alpha(q) = \frac{ 1- \sum_{f \in F_v} H^\alpha(\Phi^{e_0(v,f),f})H^\alpha(\Phi^{e_1(v,f),f})
- \sum_{e \in E_v}
\left(1- H^\alpha(\Phi^{e,f_0(e)}-H^\alpha(\Phi^{e,f_1(e)})H^\alpha(\Phi^{v,e}\right)
}{(r^v)^p}
\]

The total potential is computed as the sum over faces, edges, and vertices: 

\[
I_\alpha(q) = \sum_{v \in V} I^v(q) +  \sum_{e \in E} I^e(q)  +  \sum_{f \in F} I^f(q) 
\]



\paragraph{Sidedness checks.} This is similar to the $\Phi^e$ terms in GCP, however, we do not need to smooth these checks out or differentiate them, as these just pick one disjoint component of the potential support for each element (face, edge, or vertex). 

In more detail, for a closed surface, the support of vertex,edge and face terms of the potential consist of two  disjoint parts. To avoid spurious forces, e.g., the case of thin surface layers, we need to eliminate the internal one. 
Note the following importnt subtlety: these are \emph{not} parts globally inside and and outside the surface 
(for self-interaction we do need the potential of one part to affect other parts, i.e., they cannot vanish exactly a the surface). 
Rather, this is a local property, determined by the neigborhood of a 
point.  In the case of PL surfaces, we define these local sidedeness checks as follows. 
\begin{itemize}
\item For \emph{faces}, let $p_0$ be a  vertex of a face $f$, and assume that  $n$ is the outward normal.  
Then the condition for the point $q$  to be on the visible side of face $f$ is  
\[
Out^f(q) = (q - p_0) \cdot n > 0
\]
if this is not satisfied, the face potential is set to zero.
\item For \emph{edges}, we consider two half-planes determined by the faces incident at the edge.  The shared boundary line of the half-planes is the line of the edge, and the half-planes contain the faces $f_0$ and $f_1$. The union of these halfplanes $C$ splits the space into two parts. Then we say that the point $q$ is \emph{locally outside} with respect to the edge $e$ if it is in the same part as the outward normals of the faces are pointing to.  Computationally this can be determined using the following approach.  Let $q^p$ be the closest point on $C$
to $q$.  It is either a projection to one of the planes of the faces, or the projection to the edge.  We only need to consider projections to the faces if these fall inside the halfplanes. To determine if the point $q$ is outside we look at the sign of the dot product associated with the corresponding halfplane or edge:  $n_0$, $n_1$, or $n_a$, where
$n^a$ be the average of the normals $n_0$ and $n_1$, which we associate with the edge.  

More specifically, for the edge we use the following check. 

Let $\Phi^{e,f_i}(q)$ $i=0,1$ be defined 
as above, for two incident faces $f_0$ and $f_1$ of an edge $e$, 
and let $r_{f_0}(q)$ and $r_{f_1}$ be the signed distances to 
the planes of the two faces. 
(If there is only one incident face, then all points are considered to be outside $e$.)  Let $r_e$ be the distance to the line of the edge.  
Define the set of candidate distances as
\[
R(q) = \{|r_{f_i}|\, \left| \Phi^{e,f_i} > 0 \right. \} \cup \{ r_e \}
\]
$R$ contains from one ($r_e$) to 3 distances, depending on whether the projection of $q$ to face planes is inside or outside each halfplane. 

Set $r = \min R$, is the distance to $C$. Let $a$ be one of $e$, $f_0$ and $f_1$, the element for which $r = r_a$. 
Then the condition for $q$ to be outside is 
\[
\mathrm{Out}^e(q) = 
\begin{cases}
r_{f^*} > 0,\; \mbox{if $a = f^*$},\\
(q - P_e(q))\cdot n_e > 0,\; \mbox{if $a = e$.}
\end{cases}
\]
\item For \emph{vertices} The condition is a bit more complicated, 
as described in the GCP paper. The approach is similar to the approach for edges.  For vertices locally, the surface is defined by planar sectors, each bounded by two rays along edges meeting at the vertex, forming a cone $C$. 
We proceed similarly:  we determine which element 
(a sector interior, a ray, or the vertex itself) contains the closest point $q_p$ to $C$, and check if the direction $q - q_p$ points to the same side as the associated outward normal. 

Let $p_v$ be the vertex position, and $p_0 \ldots p_{k-1}$ be positions of the adjacent vertices. 
Let $f_0, \ldots f_{k-1}$ be faces incident at vertex $v$, and $e_0,\ldots e_{k-1}$ be edges $(p_v,p_i)$, so that $e_i$ is shared between 
$f_{i-1}$ and $f_i$, with $f_{-1}$ defined to be $f_{k-1}$. 
Let $l_i$ be the rays corresponding to these edges, and 
$w_i = (p_v-p_i)_+$ be vectors along the edges. 
Recall that $\Phi^{v,e}(q) = w_i \cdot (q - p_v)$, 
and $\Phi^{e_i,f_i}$ and $\Phi^{e_{i+1},f_i}$ are dot products of the direction towards the interior of the face $f_i$ with $q - P_{e_i}$. 

Define 
\[
R_f(q) = \{|r_{f_i}| \left| \Phi^{e_i, f_i}(q) > 0\,\mbox{and}\, \Phi^{e_{i+1},f_i}(q) > 0 \right. \}
\]
where $e_{k} = e_{0}$, and let $r_{f^*} = \min R_f$,  if $R_f$ is not empty,  and let $f^*$ be the face for which  the minimum is achieved. 
or  set $r_{f^*} = \infty$ otherwise.   

Define 
\[
R_e(q) = \{|r_{e_i}| \left| \Phi^{v, e_i}(q) > 0 \right.\}
\]
i.e., the set of all edge rays for which $q$ projects inside.
If $R_e$ is not empty, then $r_{e^*} = \min R_e$, and 
$e^*$ is the edge for which the minimum is achieved. Otherwise 
set $r_{e^*}= \infty$. 
Let $r = \min (r_{f^*}, r_{e^*}, r_v)$, and $a(q) = f^*$, $e^*$, or $v$,
depending on which element has the minimal distance value.
For each  edge $e_i$, define $n_{e_i}$ as the average of outward normals  $n_{i-1}$ and $n_i$, of incident faces $f_{i-1}$ and $f_i$. 


If the closest point is at the vertex $v$, this case is surprisingly tricky to deal with.  All such points (which may not exist for some $v$) are inside a \emph{polar cone} of the cone $C$, i.e., set of points for which $(q-p_v) \cdot e_i < 0$ 
for all $e_i$.  All such points are either outside or inside, depending on the normal orientation on faces of $C$, 
so the classification, once we know that the closest point on $C$ is $v$, does not depend on the position of $q$, and can be precomputed for each vertex.  Note that this is not a simple matter of convex vs. non-convex vertices: no-convex 
vertices may have non-empty polar cones. We denote the boolean value indicating whether the normal cone is outside or inside relative to a vertex $v$ by $PC(v)$.   Computing this quantity in full generality is nontrivial. 

\[
\mbox{Out}^{v}(q)=
\begin{cases}
r_{f^*} > 0,\; \mbox{if $a(q) = f^*$,}\\
(q - P_{e^*}(q))\cdot n^{e^*} > 0,\; \mbox{if $a(q) = e^*$.}\\
(q - p_v) \cdot (p_{j(q)} - p_v) < 0,\; \mbox{if $a(q) = v$ and 
there is a $j(q)$, s.t., $(q - p_v) \cdot (p_{j(q)} - p_v) \neq 0$ }\\
PC(v),\; \mbox{if $a(q) =v$} 
\end{cases}
\]
\end{itemize}


\section{Simplified potential definition}

In this case, we define the potential in exactly the same way for vertices and edges; only the distance is computed: 
\[
P_{v}(q) = \frac{h_\epsilon(r_v(q))}{r_v(q)^p},
\; 
P_{e}(q) = \frac{h_\epsilon(r_e(q))}{r_e(q)^p}.
\]

For a face, we use  also use:

\[
P_{f}(q) = \frac{h_\epsilon(r_f(q))}{r_f(q)^p}
\]
where $r_f(q)$ is the distance from $q$ to the face. 
The formulas for the vertex and edge are same as in 2D: 

The total potential is defined as follows: 

\begin{equation}
P(q) = \sum_f P_{f}(q) - \sum_e (\mbox{val}(e)-1)P_e(q) 
- \sum_v (\mbox{val}(v)-1)P_v(q) 
\label{eq:simplified-3D}
\end{equation}
where the vertex valence $\mbox{val}(v)$ is defined as the number of incident edges, and the edge valence $\mbox{val}(e)$ is the number of incident faces (1 or 2).

\paragraph{Computing $r_e$ and $r_f$} To compute the distance from a point $q$ to an edge $e$, 
with endpoints $p_0$ and $p_1$, we compute the projection $q_e$ of $q$ to the line of the edge, and check if it is inside the segment, if yes, then the distance $r_e = \| q - q_e\|$.  If $q_p$ is outside the segment, then $r_e$ is the distance to the closest endpoint.   

For a face $f$, consider the projection  $q_f$ of $q$ onto the plane of the face. If it is in the face interior, then $r_f =\| q -q_f\|$.  Otherwise, project it to the lines of edges to get projections $q_{e_i}$, and determine in which cases it is inside the edge segments $e_i$, $i= 0\ldots 2$. If there is one or more such edge, take the minimal distance $\| q - q_{e_i}\|$. Finally, if the projections to all edges are outside the segments, take the distance   $\| q - p_i\|$, $i=0\ldots 2$ to the closest vertex of the face. 

\end{document}